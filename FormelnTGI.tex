\documentclass[8pt]{extarticle}
 % TODO Minimumschaltung + Funktion drauf

\usepackage{graphicx}
\usepackage[utf8]{inputenc}
\usepackage{amsmath}
\usepackage[top=1in, bottom=1.25in, left=1.25cm, right=1.25in]{geometry}


\begin{document}

\begin{minipage}{0.33\textwidth}

\underline{\textbf{Grundformeln:}}\\
WS: $R = \frac{U}{I} = \frac{\rho \cdot l}{A}$; $[R] = 1\frac{V}{A} =1 \Omega$\\
Leitwert: $G = \frac{1}{R}$\\
\textbf{Maschensatz: $\mathbf{\sum U_i = 0}$\\}
\textbf{Knotensatz: $\mathbf{\sum I_i = 0}$\\}
Reihe WS: $R_{ges} = \sum R_i$\\
Parallel WS: $\frac{1}{R_{ges}} = \sum \frac{1}{R_i}$\\
\phantom{ss} Spezialfall: $R_{ges} = \frac{R_1 \cdot R_2}{R1+R2} $\\
\textbf{U-Teiler: $\mathbf{\frac{U_1}{U_2} = \frac{R_1}{R_2}\Leftrightarrow I_1 = I_2}$\\}
\textbf{I-Teiler: $\mathbf{\frac{I_1}{I_2} = \frac{R_2}{R_1}\Leftrightarrow U_1 = U_2}$\\}
Leistung: $P =U \cdot I = \frac{U^2}{R} = I^2 \cdot R $ \\
\phantom{ssssssssss} $[P] = 1V \cdot A =1 W$\\
Kein Strom $\Rightarrow$ U in Masche gleich!\\
U zu Masse $\Rightarrow$ alles danach drin

\underline{\textbf{Kondensator:}}\\
Kapazität: $C = \frac{Q}{U}, [C]=F=\frac{A \cdot s}{V}$\\
Plattenkondensator: $C=\frac{\varepsilon \cdot A }{d}$\\
Ladung: $Q = \int_{t_0}^t i(\tau) d\tau + q_0$\\
\phantom{sssssssis} $\stackrel{homogen}{=} I \cdot t$\\
\phantom{ssssssssssii} $[q]=C=A \cdot s$\\
Spannung: $U = \frac{Q}{C}$\\
Kond. parallel: $C_{ges} = C_1 + C_2$\\
Kond. in Reihe: $C_{ges} = C_1 || C_2$\\
Kond. in Reihe: $Q_1 = Q_2 = Q_{12}$\\
$u_c(t) = u_c(\infty) + (u_{pre} - u_c(\infty)) \cdot e^{-\frac{t-t_0}{\tau}}$\\
Str.q: $U_c(t) = \frac{Q}{C} = \frac{I_q \cdot (t-t_0)}{C} + U_c(t_0)$\\
\phantom{sss} ,Kennlinie linear\\
Zeitkonstante: $\tau = R \cdot C;[\tau] = 1s$\\
Zeit Auf-/Entlanden: $t = 3\tau$\\
Paralleler $R_1$: $\tau = (R_1 || R_2) \cdot C$\\
\includegraphics[scale=0.15]{kondensatorpr.png}\\
$t =\frac{T}{2} = \frac{1}{2f}$\\
"einge." $\Rightarrow i_C = C \cdot \frac{dU_c}{dt} = 0$\\
\phantom{sss} $\Rightarrow U_x = 0 \Rightarrow U_Y = U_B$

Arbeit(=Energie): $W = \int_0^\infty P dt$\\
\phantom{sssssssss} $= \frac{1}{2} U_b^2 \cdot C$ \\
\phantom{sssssssss} $= [W] = J = V \cdot A \cdot s$\\
$W_{Zyklus} = W_a + W_e$\\
$\overline{P} = \frac{W_{Zyklus}}{T}$\\
\textbf{Tiefpass}: Integrator\\
\includegraphics[scale=0.3]{tiefpass.png}\\
\includegraphics[scale=0.1]{tiefpassuit.png}
\end{minipage}%
~~~~~~~
\begin{minipage}{0.33\textwidth}
\textbf{Hochpass:} Differentiator\\
\includegraphics[scale=0.3]{hochpass.png}\\
\includegraphics[scale=0.1]{hochpassuit.png}\\
\underline{\textbf{Felder:}}\\
Stromdichte: $S=\frac{I}{A}$; $[S] = \frac{A}{m^2}$\\
Feldstärke: $E=\frac{S}{\kappa}$\\
Leitfähigkeit $\kappa = \rho \cdot \mu$\\
Ladungstr.bewegl.: $\rho$; Raumlad.dicht.: $\mu$\\
Homogenes Feld: $E=\frac{U}{s}$ {\tiny$s\hat{=}$ Plattenabstand}\\
Spannungsabfall: $U=E \cdot l$\\
Pot.: $\varphi (x)= \int_x^0 E(s)$ds$ + \varphi(0)$\\
Stromstärke: $I = \oint H $ds$ = H \cdot 2\pi d$\\
Mag.Feldstr.: $H = \frac{I}{2\pi d}, [H] = 1\frac{A}{m}$\\
\includegraphics[scale=0.14]{mfeld.png}\\
Kraft pro Leitungslänge:\\
\phantom{ss} $F=Q \cdot v \cdot B, [F] = 1N$\\
Gerader Leiter: $F=I \cdot l \cdot B$ {\tiny($\vec{B} \bot \vec{I}$)}\\
\phantom{ss}($l \hat{=}$Länge d. Leiters)\\
Ladung im Raum: $Q = \rho \cdot V$\\
$e^-$, Drift-Gesch.: $v = \frac{S}{\rho}, [v] = 1\frac{m}{s}$\\
Flussdichte: $B=\mu \cdot H, [B] = 1 \frac{Vs}{m^2}$\\
 
\underline{\textbf{Leitungen}}:\\
$R_W = \sqrt{\frac{L'}{C'}} = \frac{\sqrt{\epsilon_r \mu_r} \cdot s}{c_0 \cdot C}$ \\
$C' = \frac{C}{l}$\\
$L' = \frac{1}{c_m \cdot \sqrt{C'}}$\\
$c_m= \frac{1}{\sqrt{L' \cdot C'}} = \frac{c_0}{\sqrt{\epsilon_r \cdot \mu_s}}$\\
$X' = \frac{X}{l}$\\

$k_0 = \frac{R_w}{R_i + R_w} $\\
$r_a = \frac{R_a - R_w}{R_a + R_w} $\\
$k_a = 1 + r_a$\\
$R_w \hat{=}$ WS der Leitung\\
$R_i \hat{=}$ WS vor der Leitung\\
$R_a \hat{=}$ WS nach der Leitung\\
$t_i \hat{=}$ High-Zeit / Halbperiode\\
Einkopplung:\\
\phantom{ss} $u_h(x,t)= k_0 \cdot u_{q}(t-\Delta t)$\\
\phantom{sss}$\Delta t = \frac{l}{c} = \frac{l}{\lambda f}$\\
Reflexion: $u_r(x,t) = r_a \cdot u_h(x,t)$\\
Allgemein:\\
\phantom{ss} $u(x,t) = u_h(x,t) + u_r(x,t)$\\
\phantom{sssssssiss} $= k_a \cdot u_h(x,t)$\\

LW-Leitung: $\alpha = \frac{L_P}{l}; [\alpha] = 1 \frac{dB}{m}$\\
Lichtleistung: $L_P = 10dB \cdot \lg (\frac{P_e}{P_a})$\\
\phantom{sssssississsssss} $L_P = l \cdot \alpha$\\
\phantom{sssssississsssss}$ [L_P] = 1dB$\\



\end{minipage}%
~~~~~~
\begin{minipage}{0.33\textwidth}
\underline{\textbf{Operationsverstärker:}}\\
Eingangs-WD $R_+ = R_- = \infty$ \\
Ausgangs-WD $R_a = 0$ \\
$\implies I_+ = I_- = 0$ \\

\includegraphics[scale=0.40]{NIOV.png}
\includegraphics[scale=0.40]{IOV.png} \\
$U_D = 0$, Gegenkoppung\\

\includegraphics[scale=0.40]{NISTOV.png}
\includegraphics[scale=0.40]{ISTOV.png}\\
$U_D \neq 0$, Mitkopplung \\
Umschaltung bei:\\
\phantom{sssi}$U_D = 0 \Rightarrow U_G = U_{SP}$\\
$U_A \in \{ U_{A,min}, U_{A,max} \}$\\
$U_D = 0 = U_E - U_{E,HL}$\\
\phantom{sssi}$\Rightarrow U_{E,HL} = \frac{R_1}{R_1 + R_2} \cdot U_{A,max}$\\
$U_D = 0 = U_E - U_{E,LH}$\\
\phantom{sssi}$\Rightarrow U_{E,LH} = \frac{R_1}{R_1 + R_2} \cdot U_{A,min}$\\
Diode: $\Rightarrow U_{E,LH} = 0$\\
Schalthysterese:\\
\phantom{sssi} $H = |U_{E,HL} - U_{E,LH}|$\\
Verstärkungsfaktor $V_D \to \infty$ \\
\phantom{sssi}$U_A = V_D \cdot U_D$ \\

\begin{tabular}{| l l l |}
	\hline
	Name & Präfix & Potenz\\[0.5ex]
	\hline
	\hline
	Peta & $P$ & $10^{15}$\\
	\hline
	Tera & $T$ & $10^{12}$\\
	\hline
	Giga & $G$ & $10^9$\\
	\hline
	Mega & $M$ & $10^6$\\
	\hline
	Kilo & $k$ & $10^3$\\[0.5ex]
	\hline
	\hline
	Milli & $m$ & $10^{-3}$\\
	\hline
	Mikro & $\mu$ & $10^{-6}$\\
	\hline
	Nano & $n$ & $10^{-9}$\\
	\hline
	Piko & $p$ & $10^{-12}$\\
	\hline
	Femto & $f$ & $10^{-15}$\\
	\hline
\end{tabular}

\end{minipage}%

\newpage

\begin{minipage}{0.33\textwidth}
\vspace{-1cm}
\includegraphics[scale=0.40]{Mult.png}\\
\includegraphics[scale=0.40]{FF.png}\\

\underline{\textbf{Diode:}}\\
Sperrt bei: $U_D < U_{F_0}$\\
\phantom{sssi}$\implies I_D = -I_S$ oder $0$\\
Leitet bei: $U_D = U_{F_0}$\\
WS-frei \\

\underline{\textbf{Transistor:}}\\
$I_G \stackrel{!}{=} 0$\\

\textbf{n-Kanal:}\\
- Pfeil auf Gate\\
- Sperrt bei: $U_{GS} < U_{th}$\\
\phantom{sssi} $\implies I_{DS} = 0$\\
- Leitet bei: $U_{GS} \geq U_{th}$ \\
\phantom{sssi} $\implies U_{DS} \geq 0, \hspace{2mm} I_{DS} \geq 0$\\
%
\textbf{p-Kanal:}\\
- Pfeil von Gate weg\\
- Sperrt bei: $U_{GS} > U_{th}$\\
\phantom{sssi} $\implies I_{DS} = 0$\\
- Leitet bei: $U_{GS} \leq U_{th}$ \\
\phantom{sssi} $\implies U_{DS} \leq 0, \hspace{2mm} I_{DS} \leq 0$
%
\[
  I_{DS,N} = \left\{
  \begin{array}{lll}
        0 & U_{GS} < U_{th} &$sperrt$ \\
        \frac{1}{2} \beta \cdot(U_{GS} - U_{th})^2 & U_{DS} > U_{GS} - U_{th} & $pinch-off$ \\
        \beta \cdot((U_{GS} - U_{th}) \cdot U_{DS} - \frac{1}{2} U_{DS}^2) & U_{DS} \leq U_{GS} - U_{th} & $aktiv$\\
  \end{array}
\right.
\]
\[
  I_{DS,P} = \left\{
  \begin{array}{lll}
        0 & U_{GS} > U_{th} &$sperrt$ \\
        -\frac{1}{2} \beta \cdot(U_{GS} - U_{th})^2 & U_{DS} < U_{GS} - U_{th} & $pinch-off$ \\
        -\beta \cdot((U_{GS} - U_{th}) \cdot U_{DS} - \frac{1}{2} U_{DS}^2) & U_{DS} \geq U_{GS} - U_{th} & $aktiv$\\
  \end{array}
\right.
\]
Mit $I_D = \frac{U}{R}$ gleichsetzen um $U_{DS}$\\
auszurechnen. \\

\underline{\textbf{CMOS:}}\\
\begin{tabular}{l|ll}
    Gate & n-Kanal & p-Kanal \\
    \hline
    L & sperrt & leitet \\
    H & leitet & sperrt
\end{tabular}
\\
\textbf{NMOS:} $\overline{y} = a(b+c)$ \\
- Nur nicht-negierte Literale, y negiert \\
- WS oben bei $U_b$ im Schaltkreis\\
\textbf{PMOS:} $y = \overline{a}(\overline{b} + \overline{c})$ \\
- Nur negierte Literale \\
- WD unten bei Masse im Schaltkreis\\
\textbf{CMOS:} PMOS und komplementärer NMOS-Teil\\
- Parallelschaltung $\implies +$ \\
- Reihenschaltung \hspace{0.1mm} $\implies \mathbf{\cdot}$ \\

\end{minipage}
~~~~~~~
\begin{minipage}{0.33\textwidth}


\underline{\textbf{Logik:}}\\
KDNF: Mint.(=1) $\Rightarrow e1e2e3 + ...\overline{0}$\\
KKNF:Maxt.(=0) $, (a+b+c) \cdot ...\overline{1}$\\
KV-Diagramme:\\ 
\textbf{3,4 Block Drehen}\\
Gray C., Rechtecke, übers. OK.\\
Gray C:$(\overline{A} \overline{B}), (\overline{A}B), (AB), (A\overline{B})$\\
min. DNF: 1en Verbinden\\
min. KNF: 0en, alle Var neg.\\
logische Fkt.:$F = \overline{AB} \Rightarrow $ NAND
log./Schalt-Verknüpfung:$y = \overline{AB} $\\
Nur NAND $\Rightarrow$ 2 Nega., keine +, 1N über ganzer Formel\\
$\overline{y}$ = NMOS\\
$\overline{NMOS} \rightarrow$ y = PMOS\\


\underline{\textbf{Multiplexer:}}\\
OBDD: Nutze Shannon(0en 1en\\
Ebenef(x)=Eb.Baum=Eb. Mult.\\

\underline{\textbf{Flip-Flop:}}\\
Zeitsteuerung:\\ ((1)$|$(2)takt(zustands)$|$(flanken))\\$|$(N. takt) - gesteuertes FF\\
$>$ Takt$\Rightarrow$ steig. TF\\
$\rlap{o}{>}$ Takt$\Rightarrow$ fall. TF\\
Datensteuerung:\\ Delay, Toggle(1E), RS, JK (2E)\\
MS-FF: $c_M=0,c_S=1 \Rightarrow$ +Fl.\\
J=S; K=R; 1,1 = Inv.\\
NAND: J-K:4, R-S:2,\\
Trans.: NAND: 4, D-FF: 14\\

\end{minipage}
~~~~~~~
\begin{minipage}{0.33\textwidth}

\underline{\textbf{Arten von Automaten:}}\\
- Mealy: \textbf{A}usgang hängt von \textbf{E}ingang und Zustand \textbf{Q} ab \\
- Moore: A hängt nur von Q ab \\
- Autonom: Kein E \\
- Asynchron: Kein Takt \\
- Flankengest. Oft Master-Slave \\

\underline{\textbf{Synthese}}\\
- Zustandsgraph $\Rightarrow$ ZF-AFT: Kenne Q, Q+ $\Rightarrow$ Wie FF's schalten

-11.6, 11.7: Graph $\Rightarrow$ Kodierung Zustände $\Rightarrow$ ZT+AT $\Rightarrow$ Schaltung
\vspace{5cm}

\underline{\textbf{Zustandsgraph:}}\\
- Ist im Prinzip totaler DFA\\
- Evnt. Startzustand\\
- K's(BZ-Code E's) als Zustände\\
- Code getakt. Ausgänge als BZ\\
- schreibe Code an jeden Zustand\\
- Ungetakt. Ausgänge an Pfeile\\
- Belegung der Eingänge an Pfeil\\

\underline{\textbf{One Hot Kodierung:}}\\
Für jeden Zustand genau ein FF an. Falls (0,0,0,...) Startzustand, dann muss FF0 immer 1 sein, außer wenn Zustand 0 ist, dann ist alles 0
\end{minipage}%

\vspace*{-0.3 cm}
\underline{\textbf{Arten von Tabellen:}}\\
- WW-Tabelle hat Eingänge und Ausgänge berechnet\\
- ZF-Tabelle hat Eingänge, Zustand(Q) $|$ und Folgezustand (Q+)\\
- ZÜ-Tabelle hat Zustand $\rightarrow$ Folgezustand(Q+) $|$ Eingänge\\
-ZF+AT: Code der Q's ohne E(K),E's, Zustände(Q), FF-E(J,K) $|$, FolgeZ(Q+), Ausgänge\textbf{!Gegenw.!},K+\\
\phantom{ss} J,K aus Formeln, Q+ aus J,K


\underline{\textbf{Misc}}
Kreis: $A = \pi \cdot r^2$\\
\end{document}
